\chapter*{Información general del curso}

\section{Descripción}
Las tecnologías de consumo, portables o de escritorio, como también las plantas industriales, requieren de electrónica digital para el manejo, procesamiento y transmisión de información. Pero el diseño y puesta a punto de esta electrónica digital implica el conocer una nueva forma de ver los circuitos y sistemas electrónicos: se conciben desde una óptica discreta y lógica según Boole. El curso “Sistemas Digitales” cubre los conceptos constitutivos del análisis y diseño de circuitos digitales y sistemas digitales en general, como también persigue la optimización de estos circuitos y hacer los diseños más eficientes, desde el punto de vista del hardware, del costo computacional y del costo económico.

Libro guía:\\
\textit{Digital Design and Computer Architecture} de David Harris y Sarah Harris. Capítulos 1 al 4.

 \section{Información de contacto}
 \subsection{Profesora}
\begin{itemize}
\item Marie González Inostroza
    \item E-mail: \email{marie.gonzalez@usm.cl}
  \item Horario de consultas: Lunes/jueves 13.00 a 14.00 presencial o en otros horarios online. En ambos casos, se debe agendar con anticipación.
\end{itemize}
\subsection{Ayudantes}
\begin{itemize}
    \item Catalina Ulloa 
        E-mail: \email{catalina.ulloa@sansano.usm.cl}
    \item Juan Pablo del Valle
E-mail: \email{juan.delvalle@sansano.usm.cl}
\end{itemize}

 \section{Horario}
\begin{itemize}
    \item Clase Martes: bloque 13-14 17:10 a 18:20 sala A014
  \item Clase Jueves: bloque 13-14 17:10 a 18:20 sala A015
  \item Ayudantía: por definir
\end{itemize}

 \section{Evaluaciones}
 
 Se realizarán 4 controles, un proyecto grupal y un examen final no eximible. 
 
La nota final ($NF$) será calculada de la siguiente forma:

\begin{equation}
     NF=P*0.3 + C*0.4 + E*0.3
\end{equation}

Donde $NE$ es la nota del examen, $NC$ es el promedio de los 3 mejores controles y $P$ es la nota del proyecto.\\

Se evaluará la realización de un examen recuperativo en caso de presentarse casos que lo ameriten.